%%%%%%%%%%%%%%%%%%%%%%%%%%%%%%%%%%%%%%%%%%%%%%%%%%%%%%%%%%%%%%%%%%%%%%
%
%   Abstrak
%
%%%%%%%%%%%%%%%%%%%%%%%%%%%%%%%%%%%%%%%%%%%%%%%%%%%%%%%%%%%%%%%%%%%%%%

\begin{center}
    \addcontentsline{toc}{chapter}{ABSTRAK}
    \pagestyle{fancy}
\end{center}

%---------------------------------------------------------------------

\begin{center}
    {\textbf{\MakeUppercase{\judulTA}}}
\end{center}

\vspace{5mm}

\noindent \begin{tabular}{l c l}
    \textbf{Nama}       & \textbf{:} & \textbf{\namaMahasiswa}  \\[-1mm]
    \textbf{NRP}        & \textbf{:} & \textbf{\noIndukMahasiswa}  \\[-1mm]
    \textbf{Departemen} & \textbf{:} & \textbf{\namaDepartemen}  \\[-1mm]
    \textbf{Pembimbing} & \textbf{:} & \textbf{1. \namaDosenPembimbingSatu}  \\[-1mm]
                        &            & \textbf{2. \namaDosenPembimbingDua}
\end{tabular}

%---------------------------------------------------------------------

\vspace{5mm}

\begin{center}
    \noindent {\textbf{{Abstrak}}}
\end{center}

%---------------------------------------------------------------------

% Catatan: Gunakan \singlespacing di tiap awal paragraf

{\singlespacing\indent%
Boron Nitride Heksagonal (hBN) merupakan material dua dimensi (2D) yang memiliki potensi besar dalam aplikasi elektronik, optoelektronik, dan komposit berkat kestabilan termalnya serta sifat isolator yang unik. Penelitian ini mengkaji sifat elektronik dari model lembaran tunggal hBN 2D berukuran 6$\times$6$\times$1 yang dipanaskan secara termal. Simulasi Dinamika Molekul (MD) dilakukan menggunakan LAMMPS dengan potensial ReaxFF, dengan rentang suhu antara 500 K hingga 4000 K. Evolusi struktur selama proses pemanasan dianalisis melalui Fungsi Distribusi Radial (RDF) dan Perpindahan Kuadrat Rata-Rata (MSD). Struktur akhir hasil pemanasan kemudian dikonversi menggunakan Lingkungan Simulasi Atom (ASE) agar kompatibel dengan perhitungan Teori Fungsional Kerapatan (DFT) yang dilakukan dengan Quantum ESPRESSO (QE). Analisis struktur elektronik material dilakukan melalui perhitungan Medan Konsisten Diri (SCF) dan Medan Tidak Konsisten Diri (NSCF) untuk memperoleh struktur pita, Kerapatan Keadaan (DOS), Kerapatan Keadaan yang Diproyeksikan (PDOS), serta distribusi muatan dengan mempertimbangkan efek polarisasi spin. Hasil simulasi MD menunjukkan bahwa pemanasan termal menghasilkan perubahan signifikan pada susunan atom.
}

%---------------------------------------------------------------------

\vspace{5mm}

\noindent \textbf{Kata kunci: Boron Nitride heksaginal; dinamika molekul; teori fungsional kerapatan; efek termal; sifat elektronik.} \textit{} % Kata kunci dalam bahasa Indonesia

\newpage


%%%%%%%%%%%%%%%%%%%%%%%%%%%%%%%%%%%%%%%%%%%%%%%%%%%%%%%%%%%%%%%%%%%%%%