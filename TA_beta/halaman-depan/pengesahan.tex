%%%%%%%%%%%%%%%%%%%%%%%%%%%%%%%%%%%%%%%%%%%%%%%%%%%%%%%%%%%%%%%%%%%%%%
%
%   Secara umum, informasi yang dibutuhkan pada lembar pengesahan
%   ini diambil dari file "informasi.tex"
%
%%%%%%%%%%%%%%%%%%%%%%%%%%%%%%%%%%%%%%%%%%%%%%%%%%%%%%%%%%%%%%%%%%%%%%

\begin{center}
    {\large\textbf{LEMBAR PENGESAHAN}}
    \addcontentsline{toc}{chapter}{LEMBAR PENGESAHAN}
    \pagestyle{fancy}
\end{center}

%---------------------------------------------------------------------

\begin{center}
    
    {\large\MakeUppercase{\textbf{{\judulTA}}}}

    \vspace{5mm}
        
    {\large\textbf{PROPOSAL TUGAS AKHIR}}

    \vspace{2mm}
    
    
    %
    Diajukan untuk memenuhi salah satu syarat \\ [-2mm]
    Memperoleh gelas S.Si pada \\[-2mm]
    Program Studi S-1 \\[-2mm] 
    \namaDepartemen \\[-2mm]
    \namaFakultas \\[-2mm]
    \namaUniversitas \\[-2mm]
    \namaKota \\[-2mm]

    \vspace{6mm}
    
    Oleh: 
   {\textbf{\MakeUppercase{\namaMahasiswa}}}\\
   \textbf{NRP.} {\textbf{\MakeUppercase{\noIndukMahasiswa}}}\\[20mm]

\end{center}

%---------------------------------------------------------------------

\begin{center}
Disetujui oleh Tim Penguji Proposal Tugas Akhir \\[2mm]  
\end{center}

\begin{flushleft}
\begin{tabular}{ l c l c}
    1.& &\namaDosenPembimbingSatu & Pembimbing \\
& & & \\
    2.& &\namaDosenPembimbingDua & Ko-pembimbing \\
& & & \\
\end{tabular}
\end{flushleft}

\vfill

\begin{center}
    \textbf{\namaKota,} \\[-2mm] 
    \textbf{\tanggalPengesahan}
\end{center}

\vfill

\newpage

%%%%%%%%%%%%%%%%%%%%%%%%%%%%%%%%%%%%%%%%%%%%%%%%%%%%%%%%%%%%%%%%%%%%%%