%%%%%%%%%%%%%%%%%%%%%%%%%%%%%%%%%%%%%%%%%%%%%%%%%%%%%%%%%%%%%%%%%%%%%%
% BAB PENUTUP
%=====================================================================
\renewcommand{\thechapter}{\Roman{chapter}}
\addtocontents{toc}{\vskip10pt}
\chapter{PENUTUP}
\renewcommand{\thechapter}{\arabic{chapter}}
%---------------------------------------------------------------------

\section{Kesimpulan}
\label{sec:kesimpulan}
Penelitian ini menunjukkan bahwa efek termal dan defek titik secara sinergis memodulasi sifat elektronik dan magnetik monolayer hBN.
Telah terbukti bahwa hBN murni mengalami pergeseran merah (redshift) celah pita yang normal akibat kopling elektron-fonon.
Sebaliknya, defek antisite N$_B$ menyebabkan pergeseran biru (blueshift) yang anomali, menunjukkan adanya mekanisme fisika yang terlokalisasi di sekitar defek.
Temuan paling krusial adalah induksi magnetisme $d^0$ oleh defek B$_N$, yang secara tak terduga menguat pada temperatur tinggi.
Fenomena ini mengindikasikan adanya kopling spin-fonon yang kuat, di mana distorsi kisi termal secara aktif menstabilkan keadaan magnetik.
Keandalan kuantitatif dari temuan ini, khususnya besaran momen magnetik, bergantung pada akurasi metodologi komputasi yang digunakan.

\section{Saran}
\label{sec:saran}
Untuk penelitian selanjutnya, disarankan untuk memvalidasi temuan ini dengan metode komputasi yang lebih akurat guna meningkatkan keandalan kuantitatif.
Ini mencakup penggunaan \textbf{potensial antar-atom berbasis machine-learning (MLIPs)} untuk simulasi dinamika molekuler yang lebih realistis dan penerapan \textbf{fungsional hibrid (seperti HSE06) atau metode GW} untuk perhitungan sifat elektronik yang lebih presisi.
Selain itu, analisis energi pembentukan defek serta penggunaan supercell yang lebih besar sangat direkomendasikan untuk mengkonfirmasi stabilitas termodinamika defek dan menghilangkan potensi interaksi antar citra periodik, sehingga memastikan bahwa sifat-sifat yang teramati bersifat intrinsik dan bukan artefak komputasi.