%%%%%%%%%%%%%%%%%%%%%%%%%%%%%%%%%%%%%%%%%%%%%%%%%%%%%%%%%%%%%%%%%%%%%%
% BAB PENUTUP
%=====================================================================
\renewcommand{\thechapter}{\Roman{chapter}}
\addtocontents{toc}{\vskip10pt}
\chapter{PENUTUP}
\renewcommand{\thechapter}{\arabic{chapter}}
%---------------------------------------------------------------------

\section{Kesimpulan}
\label{sec:kesimpulan}
Berdasarkan analisis komputasi multi-skala yang telah dilakukan, penelitian ini menghasilkan beberapa kesimpulan fundamental yang secara langsung menjawab rumusan masalah:
\begin{enumerate}
    \item Pengaruh temperatur pada monolayer hBN murni termanifestasi sebagai penurunan monoton pada celah pita energi (pergeseran merah), dari 4.415 eV pada 800 K menjadi 4.069 eV pada 1225 K. Fenomena ini merupakan akibat dari interaksi kopling elektron-fonon yang menguat seiring dengan meningkatnya vibrasi kisi.

    \item cacat antisite secara kualitatif mengubah respons termal sistem. cacat N$_B$ menginduksi keadaan terlokalisasi di dalam celah pita dan menyebabkan pergeseran biru (\textit{blueshift}) yang anomali, di mana celah pita efektif melebar dari 0.694 eV menjadi 1.214 eV seiring kenaikan temperatur. Sebaliknya, cacat B$_N$ menunjukkan respons termal yang berbeda secara fundamental dengan menginduksi transisi fasa magnetik.

    \item cacat antisite B$_N$ terbukti mampu menginduksi magnetisme $d^0$ pada monolayer hBN yang intrinsik non-magnetik. Secara krusial, distorsi struktural yang diinduksi oleh temperatur tidak menghancurkan, melainkan menginduksi dan memperkuat momen magnetik. Sistem yang non-magnetik pada 800 K menunjukkan magnetisasi total 1.850 $\mu_B$ pada 1225 K , mengindikasikan adanya mekanisme kopling spin-fonon yang kuat, di mana vibrasi termal menstabilkan keadaan magnetik.Keandalan kuantitatif dari temuan-temuan ini, khususnya besaran momen magnetik, bergantung pada akurasi metodologi komputasi yang digunakan dan memerlukan validasi lebih lanjut.
\end{enumerate}

\section{Saran}
\label{sec:saran}
Untuk penelitian selanjutnya, disarankan untuk memvalidasi temuan ini dengan metode komputasi yang lebih akurat guna meningkatkan keandalan kuantitatif.
Ini mencakup penggunaan potensial antar-atom berbasis \emph{machine-learning} (MLIPs) untuk simulasi dinamika molekuler yang lebih realistis dan penerapan fungsional hibrid (seperti HSE06) atau metode GW untuk perhitungan sifat elektronik yang lebih presisi.
Selain itu, analisis energi pembentukan cacat serta penggunaan supercell yang lebih besar sangat direkomendasikan untuk mengkonfirmasi stabilitas termodinamika cacat dan menghilangkan potensi interaksi antar citra periodik, sehingga memastikan bahwa sifat-sifat yang teramati bersifat intrinsik dan bukan artefak komputasi.